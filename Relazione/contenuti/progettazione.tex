\section{Progettazione}
Durante la fase di progettazione è stata effettuata una scelta fondamentale: i linguaggi da utilizzare. Per consentire una migliore integrazione con le tecnologie assistive di ultima generazione, e per sfruttare al meglio le potenzialità del design Responsive
si è scelto di utilizzare HTML5 e CSS3, in simbiosi con JavaScript e PHP5.

Un ulteriore sicurezza sulla scelta delle tecnologie da utilizzare  ci é stata fornita dal confronto con i \textit{competitors}: effettuano la nostra stessa scelta, come nel caso di \href{www.dolomitisuperski.com}{Dolomiti Superski}

Grazie all'utilizzo di CSS3 abbiamo potuto svilupare una interfaccia con l'utilizzo dei \verb|display:flex| e \verb|display:grid|.


\subsection{Attori}
Un fruitore del nostro sito web ricade sempre in una delle seguenti 3 categorie:
\subsubsection{Amministratore}
È la categoria di utenti con più potere, può visitare l'area riservata da cui gestire i prezzi degli skipass e lo stato del comprensorio.
È un account interamente dedicato all'amministrazione, non personale, e pertanto ci si aspetta un utilizzo prevalentemente da Desktop o Tablet. Per questo motivo abbiamo il link alla pagina Dashboard non è presente nella visualizzazione da Cellulare, ma è sempre raggiungibile al link noto per interventi straordinari.
\subsubsection{Utente standard}
Può usufruire di tutte le funzionalità del sito tranne la prenotazione degli skipass e l'amministrazione.
\subsubsection{Utente registrato}
Questa categoria di utenti ha la possibilità di prenotare online gli skipass, oltre a tutte quelle di un Utente standard

\subsection{Struttura del sito}

\subsection{Menù e breadcrum}

\subsection{Contenuti}

\subsection{Shop}

\subsection{Dashboard}

\subsection{Database}

\subsection{Accessibilità}