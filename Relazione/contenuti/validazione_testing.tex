\section{Validazione e testing}
Per assicurarci della compatibilità con il più ampio spettro di dispositivi abbiamo eseguito test dell'interfaccia e del funzionamento su tutti i browser e sistemi operativi a nostra disposizione: Google Chrome, Mozilla Firefox sono stati testati su Windows e Ubuntu, Edge e Internet Explorer su Windows, Google Chrome su android.
Il supporto per il browser IE non era inizialmente previsto, ma è stato per quanto possibile reso disponibile: il sito ha alcuni difetti grafici ma funziona tuttavia correttamente. Questo problema è trascurabile data lo scarso share di mercato posseduto dal browser e il suo largamente preannunciato \textit{retirement}.


\subsection{Validazione}
Per la validazione del pagine web, è stato utilizzato \href{https://validator.w3.org/}{il validatore ufficiale di w3c}.
Allo stesso modo, per la validazione dei fogli di stile è stato utilizzato \href{https://jigsaw.w3.org/css-validator/}{il validatore ufficiale di w3c}


Alcuni errori vengono segnalati dal validatore CSS: questi riguardano alcune proprietà utilizzate per massimizzare la retrocompatibilità, e si è deciso di trascurarli per questo motivo.

\subsection{Testing}
Il testing è stato effettuato con lo scopo di rilevare e migliorare le funzionalità e l'accessibilità del sito web.
Il testing ha compreso:
\begin{itemize}
    \item Utilizzo proprio ed improprio dei punti di interazione con il backend (Login, Registrazione, Shop e Dashboard)
    \item Controllo della trasformazione elegante (a scopo di verificare la compatibilità con diverse dimensioni di schermi)
    \item Controllo dei contrasti
    \item Controllo dell'adesione alle regole WCAG
    \item Test di Drue Miller con utilizzo di un simulatore di screen reader
\end{itemize}

Alcuni di questi controlli sono stati eseguiti manualmente, mentre altri sono stati eseguiti con l'ausilio della modalità sviluppatore o di estensioni per il browser.
Le estensioni utilizzate sono: \begin{itemize}
    \item \href{https://addons.mozilla.org/en-US/firefox/addon/ainspector-wcag/}{AInspector WCAG}
    \item \href{https://addons.mozilla.org/en-US/firefox/addon/wcag-contrast-checker/}{WCAG Color contrast checker}
    \item Inserite quelle che avete utilizzato voi
\end{itemize}