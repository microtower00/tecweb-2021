\section{Realizzazione}

\subsection{Divisione dei compiti}

La realizzazione del progetto è stata così suddivisa tra i membri del gruppo:
\begin{itemize}
    \item Cazzaro Michele:
        \begin{itemize}
            \item Interfaccia di registrazione e login (pagine "Registrati", "Login", "Logout");
            \item Definizione della tabella Utenti e stesura iniziale del DB;
            \item Creazione del layout flex per gestire il menu comprimibile su tablet;
            \item Stesura dei contenuti di: "Home", "Chi Siamo", "Come Raggiungerci";
            \item Definizione di un file con metodi di utility (utils.php);
            \item Aggiunte per la navigazione accessibile (Skip nav e Scroll top);
            \item Contributo alla relazione;
        \end{itemize}
    \item Contin Riccardo:
        \begin{itemize}
            \item Impostazione iniziale menù, breadcrumb e "Home";
            \item Pagine HTML/PHP: "Il nostro comprensorio", "Mappa delle piste", "Dashboard Admin", "Modifica Comprensorio";
            \item PHP: dbRicky.php;
            \item JS: modificaComprensorio.js e js per la mappa;
            \item CSS: lavorato su style.css, print.css e mobile.css;
            \item Scrittura della relazione.
        \end{itemize}
    \item Speranzon Leonardo:
        \begin{itemize}
            \item Piccole modifiche al DB: tabelle Carrello, Oridini e SkipassOrdinati;
            \item Pagine HTML/PHP: "Shop", "Carrelo", "Modifica Prezzi Skipass";
            \item JS: shop.js per la gestione del form dello shop;
            \item CSS: creato print.css e adattato style.css ad IE10;
            \item Contributo alla relazione
        \end{itemize}
\end{itemize}

\subsection{Struttura}
    Per la struttura è stato utilizzato HTML5, rispettando sempre la sintassi XML.
    All'interno dei file .html sono presenti dei placeholder del tipo \verb|['NomePlaceholder']| che vengono sostituiti dal php per inserire elementi ripetitivi tra le varie pagine o creati dinamicamente.
    Alcuni esempi di placeholder possono essere:
    \begin{itemize}
        \item \verb|['Imports']| viene sostituito dai vari link ai fogli di stile con le relative media query ed dallo script per il funzionamento del menu;
        \item \verb|['Menu']| viene sostituito da un menu costruito dinamicamente in base alla pagine corrente e dai privilegi dell'utente;
        \item \verb|['BtnSkip']| e \verb|['BtnScroll']| vengono sostituiti dai bottoni per agevolare la navigazione del sito.
    \end{itemize}
    Tutti i file HTML sono stati inseriti all'interno della cartella html.

\subsection{Presentazione}
\subsubsection{Desktop}
\subsubsection{Tablet}
\subsubsection{Cellulari}
\subsubsection{Stampa}
    
    
\subsection{Comportamento}