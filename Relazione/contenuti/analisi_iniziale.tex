\section{Analisi iniziale}

\subsection{Analisi degli utenti}

\subsubsection{Sciare}
Sciare è una passione che accomuna moltissime persone che aspettano con trepidazione la stagione invernale
per poter godersi giornate sulla neve con l'attrezzatura che preferiscono. Che siano gli sci o lo snowboard,
infatti, le piste sono l'attrazione principale da Dicembre a Marzo.

A condividere questa passione sono sciatori esperti, ma anche sciatori alle prime armi che magari si affacciano
per le prime volte a questo sport molto spesso accompagnati da istruttori qualificati che permettono di divertirsi
in sicurezza.

\subsubsection{Passione costosa}
Il lato negativo è che sciare è una passione costosa che non permette a tutti di praticarla frequentemente, infatti 
uno skipass giornaliero non costa meno di 50 \euro. Se poi si vanno ad aggiungere: i soldi per arrivare al comprensorio
tra autostrada e carburante, i soldi per il pranzo nei rifugi ed eventualmente del noleggio dell'attrezzatura è facile 
intuire che una giornata di sci è alquanto costosa.

\subsubsection{Categorie}
Sciare, però, resta un'attrazione amata da moltissime persone e che comprende quasi tutte le categorie, famiglie
coi bambini, giovani, adulti, ma anche, in numero ridotto, persone anziane che riescono ancora a destreggiarsi sugli sci.
Inoltre, è importante tenere in considerazione anche le persone svantaggiate: le persone non vedenti, infatti, possono sciare
se accompagnati da un'atleta guida; le persone con disabilità agli arti superiori possono sciare tranquillamente senza bastoncini o
con uno solo a seconda dei casi; le persone con disabilità agli arti inferiori possono sciare utilizzando il monosci, attrezzo composto
da un sedile montato su uno sci e con il supporto di stabilizzatori utilizzabili con gli arti superiori.

\subsubsection{Conclusioni}

In conclusione, l'accesso al nostro sito deve essere garantito a tutte le categorie di persone e deve considerare anche la diversità
dei dispositivi con cui il sito viene visualizzato.

Inoltre, poiché si sta considerando un pubblico composto da diverse categorie è importante non concentrarsi solamente sulle ultime versioni
dei sistemi operativi e dei browser, ma ricercare le possibili versioni precedenti ancora utilizzate da una percentuale consistente per rendere
accessibile il sito anche su quest'ultime.

\subsection{Analisi per un buon posizionamento}

Per fare in modo che il sito possa ottenere un buon posizionamento sui motori di ricerca, abbiamo deciso di procedere nel seguente modo:
\begin{itemize}
    \item Analizzare i termini più utilizzati sui motori di ricerca in relazione a quello che il sito vuole trattare;
    \item Analizzare i principali concorrenti prendendo in considerazione le keywords da loro utilizzate;
    \item Stilare una lista di \textit{keywords} idonee al nostro sito.
\end{itemize}

\subsubsection{Analisi dei competitors}

Analizzando i vari competitors, abbiamo individuato in \href{www.dolomitisuperski.com}{Dolomiti Superski} quello di riferimento poichè è tra i siti di maggior 
importanza tra i comprensori sciistici ma soprattuto i comprensori che esso presenta sono geograficamente vicini al nostro.

Analizzando il loro sito, abbiamo visto che non utilizzano \textit{keywords} nei loro tag \textit{meta}. Per questo ci siamo limitati ad analizzare nei loro
tag \textit{meta} le \textit{description}.

\subsubsection{Risposta alle possibili ricerche sui motori di ricerca}

Dopo aver fatto le nostre analisi sul settore di riferimento e sui concorrenti già presenti con i loro siti, abbiamo stilato le nostre \textit{keywords} e inserite
nei tag \textit{meta} nelle pagine che vogliamo vengano indicizzate oltre ai tag \textit{meta} con attributo \textit{description}. 

Il sito si impegna, quindi, a rispondere alle seguenti possibili ricerche sui motori di ricerca:
\begin{itemize}
    \item Skipass Valle Bianca
    \item Sciare Valle Bianca
    \item Comprensorio sciistico dolomiti
    \item Migliori impianti da sci
    \item Sciare Dolomiti
\end{itemize}
