\documentclass[12pt, a4paper]{article} 

\usepackage{amsmath}
\usepackage{hyperref}
\usepackage[table,xcdraw]{xcolor}
\usepackage{float}
\usepackage[T1]{fontenc}
\usepackage[utf8]{inputenc}
\usepackage[italian]{babel}
\usepackage{fancyhdr}
\usepackage{eurosym}

\hypersetup{
    colorlinks=true,
    linkcolor=black,
    urlcolor=blue,
}

\pagestyle{fancy}
\lhead[]{}
\lfoot{Comprensorio Valle Bianca}
\cfoot{}
\rfoot{\thepage}
\renewcommand{\headrulewidth}{0.4pt}
\renewcommand{\footrulewidth}{0.4pt}
\setlength{\headheight}{15pt}


\title{Comprensorio Valle Bianca}
\author{Leonardo Speranzon, Riccardo Contin, Michele Cazzaro}
\date{\today}

\begin{document}

    \begin{titlepage}
        \begin{center}
            \vspace*{1cm}
            
            \Huge
            \textbf{Comprensorio Valle Bianca}

            \vspace{0.5cm}
            \LARGE
            Relazione del progetto di Tecnologie Web

            \vspace{1.5cm}
            \Large
            Cazzaro Michele, Contin Riccardo, Speranzon Leonardo
        \end{center}
            \vfill

            \large
            Indirizzi:
            \begin{itemize}
                \item \textbf{Indirizzo web del sito: } \url{tecweb.studenti.math.unipd.it/rcontin}
                \item \textbf{Indirizzo email del referente: } \href{mailto:riccardo.contin.2@studenti.unipd.it}{riccardo.contin.2@studenti.unipd.it}
            \end{itemize}

            \begin{table}[H]
                \centering
                \begin{tabular}{|c|c|c|}
                    \hline
                    \rowcolor[HTML]{96FFFB} 
                    \textbf{Utente} & \textbf{Username} & \textbf{Password} \\ \hline
                    Amministratore & admin & admin \\ \hline
                    Utente & user & user \\ \hline
                \end{tabular}
                \caption{Utenti}
            \end{table}
        
    \end{titlepage}

    \tableofcontents

    \newpage

    \begin{abstract}
        Quello che abbiamo realizzato è un sito web che ha come oggetto Valle Bianca, un comprensorio sciistico non realmente esistente.

        Il sito permette di conoscere la storia di chi si occupa di questo posto, ma soprattutto permette la visualizzazione delle piste e degli impianti presenti, sia in forma
        testuale che grazie ad una mappa interattiva. Inoltre, è possibile acquistare gli skipass e ottenere le informazioni necessarie per arrivare al comprensorio. 
    \end{abstract}

    \section{Analisi iniziale}

\subsection{Analisi degli utenti}

\subsubsection{Sciare}
Sciare è una passione che accomuna moltissime persone che aspettano con trepidazione la stagione invernale
per poter godersi giornate sulla neve con l'attrezzatura che preferiscono. Che siano gli sci o lo snowboard,
infatti, le piste sono l'attrazione principale da Dicembre a Marzo.

A condividere questa passione sono sciatori esperti, ma anche sciatori alle prime armi che magari si affacciano
per le prime volte a questo sport molto spesso accompagnati da istruttori qualificati che permettono di divertirsi
in sicurezza.

\subsubsection{Passione costosa}
Il lato negativo è che sciare è una passione costosa che non permette a tutti di praticarla frequentemente, infatti 
uno skipass giornaliero non costa meno di 50 \euro. Se poi si vanno ad aggiungere: i soldi per arrivare al comprensorio
tra autostrada e carburante, i soldi per il pranzo nei rifugi ed eventualmente del noleggio dell'attrezzatura è facile 
intuire che una giornata di sci è alquanto costosa.

\subsubsection{Categorie}
Sciare, però, resta un'attrazione amata da moltissime persone e che comprende quasi tutte le categorie, famiglie
coi bambini, giovani, adulti, ma anche, in numero ridotto, persone anziane che riescono ancora a destreggiarsi sugli sci.
Inoltre, è importante tenere in considerazione anche le persone svantaggiate: le persone non vedenti, infatti, possono sciare
se accompagnati da un'atleta guida; le persone con disabilità agli arti superiori possono sciare tranquillamente senza bastoncini o
con uno solo a seconda dei casi; le persone con disabilità agli arti inferiori possono sciare utilizzando il monosci, attrezzo composto
da un sedile montato su uno sci e con il supporto di stabilizzatori utilizzabili con gli arti superiori.

\subsubsection{Conclusioni}

In conclusione, l'accesso al nostro sito deve essere garantito a tutte le categorie di persone e deve considerare anche la diversità
dei dispositivi con cui il sito viene visualizzato.

Inoltre, poiché si sta considerando un pubblico composto da diverse categorie è importante non concentrarsi solamente sulle ultime versioni
dei sistemi operativi e dei browser, ma ricercare le possibili versioni precedenti ancora utilizzate da una percentuale consistente per rendere
accessibile il sito anche su quest'ultime.

\subsection{Risposta alle possibili ricerche sui motori di ricerca}

Per fare in modo che il sito possa ottenere un buon posizionamento sui motori di ricerca, abbiamo deciso di procedere nel seguente modo:
\begin{itemize}
    \item Analizzare i termini più utilizzati sui motori di ricerca in relazione a quello che il sito vuole trattare;
    \item Analizzare i principali concorrenti prendendo in considerazione le keywords da loro utilizzate;
    \item Stilare una lista di keywords idonee al nostro sito.
\end{itemize}

Il sito si impegna a rispondere alle seguenti possibili ricerche sui motori di ricerca:
\dots

    \newpage
    \section{Progettazione}
Durante la fase di progettazione è stata effettuata una scelta fondamentale: i linguaggi da utilizzare. Per consentire una migliore integrazione con le tecnologie assistive di ultima generazione, e per sfruttare al meglio le potenzialità del design Responsive
si è scelto di utilizzare HTML5 e CSS3, in simbiosi con JavaScript e PHP5.

Un ulteriore sicurezza sulla scelta delle tecnologie da utilizzare  ci é stata fornita dal confronto con i \textit{competitors}: effettuano la nostra stessa scelta, come nel caso di \href{www.dolomitisuperski.com}{Dolomiti Superski}

Grazie all'utilizzo di CSS3 abbiamo potuto svilupare una interfaccia con l'utilizzo dei \verb|display:flex| e \verb|display:grid|.


\subsection{Attori}
Un fruitore del nostro sito web ricade sempre in una delle seguenti 3 categorie:
\subsubsection{Amministratore}
È la categoria di utenti con più potere, può visitare l'area riservata da cui gestire i prezzi degli skipass e lo stato del comprensorio.
È un account interamente dedicato all'amministrazione, non personale, e pertanto ci si aspetta un utilizzo prevalentemente da Desktop o Tablet. Per questo motivo abbiamo il link alla pagina Dashboard non è presente nella visualizzazione da Cellulare, ma è sempre raggiungibile al link noto per interventi straordinari.
\subsubsection{Utente standard}
Può usufruire di tutte le funzionalità del sito tranne la prenotazione degli skipass e l'amministrazione.
\subsubsection{Utente registrato}
Questa categoria di utenti ha la possibilità di prenotare online gli skipass, oltre a tutte quelle di un Utente standard

\subsection{Struttura del sito}

\subsection{Menù e breadcrum}

\subsection{Contenuti}

\subsection{Shop}

\subsection{Dashboard}

\subsection{Database}

\subsection{Accessibilità}
    \newpage
    \section{Realizzazione}

\subsection{Divisione dei compiti}

La realizzazione del progetto è stata così suddivisa tra i membri del gruppo:
\begin{itemize}
    \item Cazzaro Michele:
        \begin{itemize}
            \item Interfaccia di registrazione e login (pagine "Registrati", "Login", "Logout");
            \item Definizione della tabella Utenti e stesura iniziale del DB;
            \item Creazione del layout flex per gestire il menu comprimibile su tablet;
            \item Stesura dei contenuti di: "Home", "Chi Siamo", "Come Raggiungerci";
            \item Definizione di un file con metodi di utility (utils.php);
            \item Aggiunte per la navigazione accessibile (Skip nav e Scroll top);
            \item Contributo alla relazione;
        \end{itemize}
    \item Contin Riccardo:
        \begin{itemize}
            \item Impostazione iniziale menù, breadcrumb e "Home";
            \item Pagine HTML/PHP: "Il nostro comprensorio", "Mappa delle piste", "Dashboard Admin", "Modifica Comprensorio";
            \item PHP: dbRicky.php;
            \item JS: modificaComprensorio.js e js per la mappa;
            \item CSS: lavorato su style.css, print.css e mobile.css;
            \item Scrittura della relazione.
        \end{itemize}
    \item Speranzon Leonardo:
        \begin{itemize}
            \item Piccole modifiche al DB: tabelle Carrello, Ordini e SkipassOrdinati;
            \item Pagine HTML/PHP: "Shop", "Carrello", "Modifica Prezzi Skipass";
            \item JS: shop.js per la gestione della form dello shop;
            \item CSS: creato print.css e adattato style.css a IE10;
            \item Contributo alla relazione
        \end{itemize}
\end{itemize}

\subsection{Struttura}
    Per la struttura è stato utilizzato HTML5, rispettando sempre la sintassi XML.
    All'interno dei file .html sono presenti dei placeholder del tipo \verb|['NomePlaceholder']| che vengono sostituiti dal php per inserire elementi ripetitivi tra le varie pagine o creati dinamicamente.
    Alcuni esempi di placeholder possono essere:
    \begin{itemize}
        \item \verb|['Imports']| viene sostituito dai vari link ai fogli di stile con le relative media query ed dallo script per il funzionamento del menu;
        \item \verb|['Menu']| viene sostituito da un menu costruito dinamicamente in base alla pagina corrente e dai privilegi dell'utente;
        \item \verb|['BtnSkip']| e \verb|['BtnScroll']| vengono sostituiti dai bottoni per agevolare la navigazione del sito.
    \end{itemize}
    Tutti i file HTML sono stati inseriti all'interno della cartella html.

\subsection{Presentazione}
Lo stile, realizzato tramite fogli di stile CSS3, è fluido e scalabile per ogni tipo di schermo.
Per questo sono stati realizzati 3 diversi CSS per gli schermi: uno stile base (per i desktop), uno aggiuntivo per i tablet ed uno per gli smartphone;
la maggiore differenza nell'uso da dispositivi diversi è il menu che varia forma e voci per adattarsi al meglio.

È stato creato anche un foglio di stile specifico per la stampa delle pagine.

\subsubsection{Desktop}
Da desktop il sito è limitato in larghezza per non richiedere all'utente un eccessivo movimento orizzontale della testa durante la lettura ed il menù è un classico menù orizzontale posizionato in alto.

\subsubsection{Tablet}
Lo stile per i tablet, o in generale per dispositivi di medie dimensioni, si applica successivamente allo stile di base e modifica solamente il menu e la conformazione della pagina home.
Il menù diventa ad hamburger e si apre verticalmente per ridurre lo spazio occupato durante la navigazione.

\subsubsection{Cellulari}
Se visualizzato da smartphone il menù diventa invece un menù composto da icone posizionate sul fondo dello schermo in modo da renderli raggiungibili anche per chi utilizza il dispositivo con una sola mano.

Viene anche modificato il layout della homepage ed alcune tabelle vengono adattate per schermi più piccoli

\subsubsection{Stampa}
Per la stampa delle pagine viene utilizzato un foglio di stile specifico che imposta un carattere più adatto alla lettura su carta e nasconde gli elementi superflui come il menu e gli aiuti alla navigazione.
La breadcrumb resta comunque visibile per dare maggiore contesto alla pagina una volta stampata.

Alcune pagine data la loro natura non sono state adattate per la stampa, queste pagine sono: shop, modificaComprensorio, modificaPrezzi e le pagine relative al processo di autenticazione.

\subsection{Comportamento}
    \newpage
    \section{Validazione e testing}

\subsection{Validazione}

\subsection{Testing}

\end{document}